\documentclass{llncs}
\usepackage{listings}
\usepackage{color}

\lstset{basicstyle=\scriptsize\sffamily}
%\lstset{language=Assembler}[avr8avra] % not standard
\lstdefinestyle{customasm}{
  morekeywords={adc,adiw,asr,brts,brcs,brhs,brne,cbr,clh,clr,clt,cp,cpi,dec,eor,ldi,lpm,lsl,nop,mov,or,rcall,ret,seh,set,subi,swap,rjmp},
  morecomment=[l][\color{blue}]{;}
}
\lstset{style=customasm}

\begin{document}

\title{Speed and size-optimized implementations of the PRESENT cipher for AVR machines}
\author{
Kostas Papagiannopoulos \\
Aram Verstegen
}
\institute{Radboud Universiteit Nijmegen}
\date{\today}

\maketitle

\abstract{
We present our progress in experimenting with the PRESENT block cipher to optimize it for the AVR architecture in both the direction of speed and area.}

\section{PRESENT}
PRESENT \cite{bogdanov2007present} is an ultra-lightweight 64-bit block cipher using 80-bit or 128-bit keys based on a substitution/permutation network.
PRESENT uses exclusive-or as a round key, a 4-bit substitution layer and a 4-bit period bit position permutation network in 31 rounds, and a final round key.
Key scheduling is a combination of bit rotation, S-box application and exclusive-OR with the round counter.

PRESENT was adopted by ISO as a standard for a lightweight block cipher. \cite{present_iso}

We chose PRESENT as a cipher to work on within a project to write and optimize AVR crypto code because we deemed it extremely well-suited for a simple microcontroller, and felt it was within our grasp to implement and optimize correctly within the allotted time.

Constructs found in PRESENT are also used in the SPONGENT \cite{bogdanov2011spongent} hash function, so the optimizations we present here might also interest SPONGENT implementors.

%The coursework asked us to draft optimized implementations in the direction of speed and code size, and these constraints also drove us to .
%by providing one faster implementation, and one smaller implementation to complement their version which strikes a balance between the two.
%We were already underway implementing the algorithm in a high-level language to understand it better when we learned of the work done by Leuven University's project group to implement PRESENT and other light-weight ciphers for AVR machines \cite{eisenbarth2012compact}.
%Although we felt it would be bad sportsmanship to peek at the code before we had some AVR code up and running, by the time we needed to implement the permutation layer our curiosity was hard to temper.
%Meanwhile we had also found 



\section{Speed optimizations}
\cite{gong2009towards}
\section{Size optimizations}

Here we will list some of the size improvements we were able to apply to the PRESENT algorithm.
While these modifications make the algorithm operate more slowly, the reduction in size would allow the cipher to be included in microcontrollers with smaller available code area.
Our version requires 205 AVR instructions (410 bytes of code) for both the encryption and decryption routines, plus two times 8 bytes for packed tables of S-box values at memory addresses \textbf{0x100} and \textbf{0x200}.
We believe this should be sufficiently small for the code to be included in an AVR machine with 1K of available Flash storage, while still allowing over half of the available area to be devoted to application-specific code.

Unfortunately, every opcode in the AVR instruction set is 16-bits wide, so there is nothing to be gained from exchanging any instructions for equivalent smaller ones (such as for example \textit{adiw Rd, 1} being more concisely expressed as \textit{inc Rd} on x86 machines).
Furthermore the AVR employs a Harvard architecture, where there is a strict separation between data and code memory; this prevents us from dynamically computing new opcodes in memory to be executed later.
Finally we note there are no `bulk' instructions which operate on several registers at once.

However, we do have access to some instructions that are specific to the AVR architecture and are uniquely suited to making parts of the code more condensed by virtue of their expressiveness, such as the \textit{swap} and \textit{cbr} instructions.

\subsection{Serialization}
The biggest size optimization we have implemented is serialization of the algorithm, keeping part of the state in SRAM while we operate on fewer registers.
This reduces the instruction count on all parts of the round update procedure except for key scheduling.

We have chosen to keep 4 bytes of state in registers at a time, as we believe it allows us to apply the permutation layer of the algorithm in software with the most size-efficiency.
Using even fewer state bytes in registers might allow for greater size reductions in other areas of the code, but applying the permutation layer may turn out to be more troublesome in that case.
%We would of course welcome efforts from the community to disprove this assumption.

\subsection{S-box packing}
\begin{figure}
\begin{lstlisting}
	unpack_sBox:
		asr ZL                ; halve input, take carry
		lpm SBOX_OUTPUT, Z    ; get S-box output
		brcs odd_unpack       ; branch depending on carry
	even_unpack:
		swap SBOX_OUTPUT      ; swap nibbles in S-box output
		rjmp unpack
        odd_unpack:
                nop                   ; guard against timing attacks
                nop
        unpack:
                cbr SBOX_OUTPUT, 0xf0 ; clear high nibble in S-box output
	\end{lstlisting}
	\caption{Unpacking bytes into nibbles in a constant cycle count takes us 8 instructions whereas unpacking unpacked nibbles takes us only 1. The net gain is only 2 bytes of code.}
\label{unpacking_code}
\end{figure}
% TODO footnotes


The PRESENT S-boxes work on 4-bit nibbles, but defining a table of nibbles in the code at first seemed less size-efficient than packing the nibbles into bytes to be unpacked.
This would save 8 bytes per S-box table to start with, but we need 5 to 8 instructions in stead of a \textbf{ld} (load) depending on whether or not we care about timing attacks to unpack the nibbles which diminishes the size benefit to 2 bytes of code. See Figure~\ref{unpacking_code}.
%We consider this a fair loss for our optimization strategy, and also see an upside: having the S-boxes unpacked as data improves throughput and makes the encryption and decryption code smaller to distribute separately.
%We decided 2 bytes of area reduction just wasn't worth the other losses.
%The packed table approach is therefore only beneficial when we require encryption and decryption routines in the target application, in which case it would save the remaining 6 bytes.

	%\footnotesize{
	%Before: \\
	%\begin{tabular}{ | c | c | c | c | c | c | c | c | c | c | c | c | c | c | c | c | }
	  %\hline                        
	  %C & 5 & 6 & B & 9 & 0 & A & D & 3 & E & F & 8 & 4 & 7 & 1 & 2 \\
	  %\hline  
	%\end{tabular}
	%\\

	%After: \\
	%\begin{tabular}{ | c | c | c | c | c | c | c | c | }
	  %\hline                        
	  %C5 & 6B & 90 & AD & 3E & F8 & 47 & 12 \\
	  %\hline  
	%\end{tabular}
	%}

\subsection{Permutation layer}
The code to apply permutations to the state in software borrows heavily from the AVR implementation of PRESENT drafted in Leuven by Eisenbarth et al \cite{eisenbarth2012compact}.
Since the permutation follows a 4-bit period in the input it seems most efficient to use 4 bytes of I/O when rotating bits off of registers into corresponding new positions.

In the Leuven implementation one bit is rolled off from 4 state registers into one output register and this block is done twice, completing one output byte. It has made ingenious use of the side-effect that both the input and the output are being rotated, allowing the operation to be reversed depending on the offset at which the procedure is called.

\subsection{Branching on SREG flags}
A subtle but real size optimization we implemented focused on the code used to drive repeated operations .
The construct in Figure~\ref{sreg_redo} allows the implementer to let a block of code be executed twice, while allowing them to take control of the machine before, after and in between these code blocks.
In our implementation this has been applied to the substitution and permutation layer procedures to save a few instructions.
This does not affect the cycle count relative to the state or input.
This construct also allows the \textit{unpack\_sBox} code we just defined to be inlined into our S-box procedure, saving 2 more instructions.

\begin{figure}
	\begin{lstlisting}
	setup_redo_block:
		clt                   ; clear T flag
		rjmp redo_block       ; do the second part
	block:
		set                   ; set T flag
		; fall through
	redo_block:
		; instructions here happen twice when called from block

		brts setup_redo_block ; redo this block? (if T flag set)
		ret
	\end{lstlisting}
	\caption{A construct that uses the state register to re-do a block twice with code executing before, after and in between, allowing more code to be inlined.}
\label{sreg_redo}
\end{figure}

We have used this construct in the procedures for both steps of applying the SP-network while retaining the property that these procedures can be called from two different offsets.

In the substitution procedure one offset substitutes only the low nibble.
The calling code applies this part, swaps the nibbles, applies it again and swaps the nibbles back.

In the permutation procedure the operations on 4 input bits is done twice, to complete one output byte.

\subsection{Keying and key scheduling}
Because we have serialized part of the algorithm our key register needs to be rotated when changing the context between steps that apply to the higher and lower 4 bytes of the cipher's state.
Because we required a procedure to rotate the key register for key scheduling anyway, that code can be reused for this purpose, and we can still apply the exclusive-or to part of the key register in the ideal position (i.e. where the bytes of the key register line up with the round counter register).

The inverse key scheduling procedure is only needed in the decryption routine, so we were able to inline it into the decryption round and combine the operations for key scheduling and rotating the register to the appropriate position for the next round.

\subsection{Reducing code size for specific applications}
While the attained size of the implementation of PRESENT should suffice for use in real-world applications, some of the input and output procedures as well as the S-box unpacking code can be omitted when only encryption or encryption is required in the application.
This is highlighted in the source code using comments.

\section{Conclusion}
	\begin{tabular}{ l c c c }
                & Reference & Speed optimized & Size optimized \\
        Size (bytes) & 936 & 1794 & 426 \\
        Encryption (cycles) & 10723 & 8721 & 90725 \\
        Decryption (cycles) & 11239 & - & 102257 \\
        RAM (bytes) & 0 & 18 & 18
	\end{tabular}

\section{References}
\bibliography{PRESENT_size_optimized_paper}{}
\bibliographystyle{plain}

\end{document}

