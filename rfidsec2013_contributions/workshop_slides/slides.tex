\documentclass{beamer}

\usepackage[english]{babel}
\usepackage{graphicx,hyperref,ru}
\usepackage{listings}
\usepackage{fancyvrb}
\usepackage{framed}
\usepackage{xcolor}
\usepackage{gnuplottex}
\lstset{basicstyle=\scriptsize\sffamily}
%\lstset{language=Assembler}[avr8avra] % not standard
\lstdefinestyle{customasm}{
  morekeywords={adc,adiw,asr,brts,brcs,brhs,brne,cbr,clh,clr,clt,cp,cpi,dec,eor,ld,ldi,lpm,lsl,mov,or,rcall,ret,seh,set,st,subi,swap,rjmp},
  morecomment=[l][\color{blue}]{;}
}
\lstset{style=customasm}
%\lstset{basicstyle=\tiny\sffamily}

% A few hacks to get references via bibtex (easy reuse for paper)
\usepackage{bibentry}
\nobibliography*
\let\newblock\relax
\setbeamertemplate{bibliography item}{}

\title[Speed and Size-Optimized PRESENT for AVR]{Speed and Size-Optimized Implementations of the PRESENT Cipher for Tiny AVR Devices}
\author[Papagiannopoulos and Verstegen]{
Kostas Papagiannopoulos \\
Aram Verstegen
}

\date{\today}

\begin{document}
\frame{\titlepage}

\begin{frame}[fragile]
\frametitle{Who We Are}
\includegraphics{ki}
\begin{itemize}
\item 2-year master programme in computer security
\item Collaboration of 3 Universities
\item Software, Hardware, Networks, Formal methods, Cryptography, Privacy, Law, Ethics, Auditing
\item \url{http://kerckhoffs-institute.org/}
\end{itemize}
\end{frame}

\begin{frame}[fragile]
\frametitle{Cryptography Engineering, Assignment 1}
\begin{quote}
``Choose and implement a block cipher on the ATtiny45 in two versions, optimized for size and speed''
\end{quote}

\begin{itemize}
\item PRESENT
\item KATAN-64
\item Klein
\item LED
\item PRINCE
\item mCrypton
\item Piccolo
\item XTEA
\item HIGHT
\end{itemize}
\end{frame}

\begin{frame}[fragile]
\frametitle{ATtiny Platform}
\begin{itemize}
\item Basic 90 AVR instructions
\item 8-bit registers
\item 32 general purpose registers
\item 4Kbytes flash
\item 256 bytes SRAM
\item 20 MHz
\end{itemize}
\end{frame}

\begin{frame}[fragile]
\frametitle{Choosing}
\begin{itemize}
\item All 64-bit SPN ciphers with 64 or 80 bit keys
\item None look impossible
\item Some based on matrix multiplication
\end{itemize}
\begin{quote}
``PRESENT looks just about perfect for the hardware and it's only 4
functions, 2 of which are tables and it's XOR based. It's my favorite
so far for sheer simplicity.'' - Aram
\end{quote}

\begin{quote}
``Yep, PRESENT seems straightforward we should definitely opt for it.
We also need a second choice, right? We could go for KATAN, it also looks ok.'' - Kostas
\end{quote}
We got our first choice.
\end{frame}


\begin{frame}[fragile]
\frametitle{PRESENT Cipher}
\includegraphics[width=\textwidth]{cipher}
\end{frame}

\begin{frame}[fragile]
\frametitle{Strategy}
\begin{tabular}{c c c }
	& Speed-optimized & Size-optimized \\
Substitution/permutation & Tabled & On-the-fly \\
Code flow & Inlined / unrolled & Re-used / looped \\
Locality & All in registers & Use more SRAM \\
Tricks & Algorithmic & Device-specific \\

\end{tabular}
\end{frame}

\begin{frame}[fragile]
\frametitle{addRoundKey}
\begin{lstlisting}
; state ^= roundkey (first 8 bytes of key register)
addRoundKey:
        eor STATE0, KEY0
        eor STATE1, KEY1
        eor STATE2, KEY2
        eor STATE3, KEY3
        eor STATE4, KEY4
        eor STATE5, KEY5
        eor STATE6, KEY6
        eor STATE7, KEY7
        ret
\end{lstlisting}
\end{frame}

\begin{frame}[fragile]
\frametitle{4-bit S-Box}
	\begin{tabular}{ | c | c | c | c | c | c | c | c | c | c | c | c | c | c | c | c | c | }
	  \hline                        
	     x & 0 & 1 & 2 & 3 & 4 & 5 & 6 & 7 & 8 & 9 & A & B & C & D & E & F \\
	  \hline                        
	  S[x] & C & 5 & 6 & B & 9 & 0 & A & D & 3 & E & F & 8 & 4 & 7 & 1 & 2 \\
	  \hline  
	\end{tabular}

\begin{itemize}
\item Original:
  \begin{lstlisting}
  .db 0x0C, 0x05, 0x06, 0x0B ...
  \end{lstlisting}
\item Compact:
  \begin{lstlisting}
  .db 0xC5, 0x6B, 0x90, 0xAD ...
  \end{lstlisting}
\item Accessing the table 4 bits at a time incurs a penalty
\item Cost to substitutes a byte: two clearings of nibbles, two lookups, two settings of nibbles, two swapping of nibbles
\item We have an 8-bit architecture, so we want to access bytes!
\end{itemize}
\end{frame}

\begin{frame}
\frametitle{Squared S-Box}
\begin{tabular}{| c | c  | c | c | c  | c  | c | c | c  | c | c | c |}
\hline
  x & 00 & 01 & 02 & 03  &  $\dots$  & 0C & 0D & 0E & 0F   \\
\hline
 S[x] & CC & C5 & C6 & CB & \dots & C4 & C7 & C1 & C2   \\
\hline
  x & 10 & 11 & 12 & 13  &  $\dots$  & 1C & 1D & 1E & 1F   \\
\hline
 S[x] & 5C & 55 & 56 & 5B & \dots & 54 & 57 & 51 & 52   \\
\hline
  \vdots & \vdots & \vdots & \vdots & \vdots  &  $\dots$  & \vdots &\vdots & \vdots & \vdots   \\
\hline
  x & F0 & F1 & F2 & F3  &  $\dots$  & FC & FD & FE & FF   \\
\hline
 S[x] & 2C & 25 & 26 & 2B & \dots & 24 & 27 & 21 & 22   \\
\hline
\end{tabular}

\begin{itemize}
\item New S-Box is 256 bytes, $16\cdot16$ combinations of 4-bits
\item It substitutes 1 byte at a time
\item No need to swap or discern high/low byte part
\item Cost to substitute a byte: only 1 lookup
\end{itemize}
\end{frame}

\begin{frame}
\frametitle{AVR Architecture}
        \begin{itemize}
        \item 16-bit opcodes
        \item Harvard architecture
        \item No bulk instructions
        \item Some SRAM
        \item SREG flags
        \item SWAP instruction
        \item CBR instruction
        \item All kinds of branching instructions
        \end{itemize}
\end{frame}

\begin{frame}
\frametitle{Optimization Strategy}
        \begin{itemize}
        \item Be concise to start with
        \item Look for repeating patterns
        \item Refactor to reuse code wherever possible
        \item Use the SRAM more
        \item Employ a different I/O pattern
        \end{itemize}
\end{frame}

\begin{frame}[fragile]
\frametitle{Serialization of the Algorithm}
\begin{lstlisting}
; state ^= roundkey (full state in registers)
addRoundKey:
        eor STATE0, KEY0
        eor STATE1, KEY1
        eor STATE2, KEY2
        eor STATE3, KEY3
        eor STATE4, KEY4
        eor STATE5, KEY5
        eor STATE6, KEY6
        eor STATE7, KEY7
        ret
\end{lstlisting}
\end{frame}

\begin{frame}[fragile]
\frametitle{Serialization of the algorithm}
\begin{lstlisting}
; state ^= roundkey (half state in registers)
addRoundKey:
        eor STATE0, KEY0
        eor STATE1, KEY1
        eor STATE2, KEY2
        eor STATE3, KEY3
        ret
\end{lstlisting}

This helps with:
\begin{itemize}
        \item doing I/O
        \item applying round keys
        \item applying S-Boxes
        \item applying P-layer
\end{itemize}

\end{frame}

\begin{frame}[fragile]
\frametitle{Indirect Register Access}
\begin{lstlisting}
; state ^= roundkey (full state in SRAM)
addRoundKey:
      clr YL               ; point Y at first key register
addRoundKey_byte:
      ld ITEMP, X          ; load input
      ld KEY_BYTE, Y+      ; load key, advance pointer
      eor ITEMP, KEY_BYTE  ; XOR
      st X+, ITEMP         ; store output, advance pointer

      cpi YL, 8            ; loop over 8 bytes
      brne addRoundKey_byte

      subi XL, 8           ; point at the start of the block
      ret
\end{lstlisting}
\end{frame}

\begin{frame}[fragile]
\frametitle{Using SREG Flags}
\begin{lstlisting}
setup_redo_block:
        clt                   ; clear T flag
        rjmp redo_block       ; do the second part
block:
        set                   ; set T flag
        ; fall through
redo_block:
        ; instructions here happen twice when called from block

        brts setup_redo_block ; redo this block? (if T flag set)
        ret
\end{lstlisting}
\end{frame}

\begin{frame}[fragile]
\frametitle{Packing S-Boxes}
	\footnotesize{
	Before: \\
	\begin{tabular}{ | c | c | c | c | c | c | c | c | c | c | c | c | c | c | c | c | }
	  \hline                        
	  C & 5 & 6 & B & 9 & 0 & A & D & 3 & E & F & 8 & 4 & 7 & 1 & 2 \\
	  \hline  
	\end{tabular}
	\\

	After: \\
	\begin{tabular}{ | c | c | c | c | c | c | c | c | }
	  \hline                        
	  C5 & 6B & 90 & AD & 3E & F8 & 47 & 12 \\
	  \hline  
	\end{tabular}
	}

\begin{lstlisting}
unpack_sBox:
        asr ZL                ; halve input, take carry
        lpm SBOX_OUTPUT, Z    ; get s-box output
        brcs odd_unpack       ; branch depending on carry
even_unpack:
        swap SBOX_OUTPUT      ; swap nibbles in s-box output
odd_unpack:
        cbr SBOX_OUTPUT, 0xf0 ; clear high nibble in s-box output
        ret
\end{lstlisting}
\end{frame}

\begin{frame}[fragile]
\frametitle{S-Box Optimization}
\begin{lstlisting}
sBoxByte:
        ; input (low nibble)
        mov ZL, ITEMP         ; load s-box input
        cbr ZL, 0xf0          ; clear high nibble in input
        rcall unpack_sBox     ; get output in SBOX_OUTPUT

        cbr ITEMP, 0xf        ; clear low nibble in output
        or ITEMP, SBOX_OUTPUT ; save low nibble to output

        ; fall through
sBoxHighNibble:
        mov ZL, ITEMP         ; load s-box input
        cbr ZL, 0xf           ; clear low nibble in input
        swap ZL               ; move high nibble to low nibble

        rcall unpack_sBox     ; get output in SBOX_OUTPUT
        swap SBOX_OUTPUT      ; move low nibble to high nibble

        cbr ITEMP, 0xf0       ; clear high nibble in output
        or ITEMP, SBOX_OUTPUT ; save high nibble to output

        ret
\end{lstlisting}
\end{frame}

\begin{frame}[fragile]
\frametitle{S-Box Optimization}
\begin{lstlisting}
sBoxHighNibble:
        clh
        swap ITEMP            ; swap nibbles
        rjmp sBoxLowNibble    ; do low nibble
sBoxByte:
        seh                   ; set H flag to re-do this block
        ; fall through
sBoxLowNibble:
        mov ZL, ITEMP         ; load s-box input
        cbr ZL, 0xf0          ; clear high nibble in s-box input

        rcall unpack_sBox

        cbr ITEMP, 0xf        ; clear low nibble in IO register
        or ITEMP, SBOX_OUTPUT ; save low nibble to IO register
        brhs sBoxHighNibble   ; do the high nibble
        swap ITEMP            ; swap nibbles
        ret
\end{lstlisting}
\end{frame}

\begin{frame}[fragile]
\frametitle{S-Box Optimization}
\begin{lstlisting}
sBoxHighNibble:
        clh
        swap ITEMP            ; swap nibbles
        rjmp sBoxLowNibble    ; do low nibble
sBoxByte:
        seh                   ; set H flag to re-do this block
sBoxLowNibble:
        mov ZL, ITEMP         ; load s-box input from IO register
        cbr ZL, 0xf0          ; clear high nibble in s-box input

unpack_sBox:
        asr ZL                ; halve input, take carry
        lpm SBOX_OUTPUT, Z    ; get s-box output
        brcs odd_unpack       ; branch depending on carry
even_unpack:
        swap SBOX_OUTPUT      ; swap nibbles in s-box output
odd_unpack:
        cbr SBOX_OUTPUT, 0xf0 ; clear high nibble in s-box output

        cbr ITEMP, 0xf        ; clear low nibble in IO register
        or ITEMP, SBOX_OUTPUT ; save low nibble to IO register
        brhs sBoxHighNibble   ; do the high nibble
        swap ITEMP            ; swap nibbles
        ret
\end{lstlisting}
\end{frame}

\begin{frame}[fragile]
\frametitle{Size Over Time}
%\includegraphics[width=\textwidth]{size_over_time}
\tiny{
\advance\leftskip-4em
\begin{gnuplot}[terminal=latex]
set xdata time
set timefmt "%b-%d"
set xrange ["Feb-23":"Jul-1"]
set yrange [250:500]
set xlabel "Time"
set ylabel "Size"
set title "Size reduction of my PRESENT implementation"
set key off
set output "size_over_time.tex"
plot "size_over_time.txt" using 1:2 index 0 title "bytes" with points, \
     ''               using 1:2:2 with labels left offset 0.8,0.2 notitle, \
     ''               using 1:2:3 with labels left offset 3,0.2 notitle

\end{gnuplot}
\include{size_over_time}
}
\end{frame}

\begin{frame}[fragile]
\frametitle{Size Per Procedure}
%\includegraphics[width=\textwidth]{procedures}
\tiny{
\advance\leftskip-4em
\begin{gnuplot}[terminal=latex]
set yrange [0:65]
set xlabel "Procedure"
set ylabel "Size"
set title "Sizes per procedure"
set key left box
unset xtics
set output "procedures.tex"
plot 'procedures.txt' using 0:2:xtic(1) title "bytes" with boxes, \
     ''          using 0:2:1 with labels offset 0,0.5 notitle
\end{gnuplot}
\include{procedures}
}
\end{frame}

\begin{frame}[fragile]
\frametitle{Numbers}
\begin{table}[h]
\centering
\footnotesize
	\begin{tabular}{ l r r r }
		& Encryption & Decryption & Size \\
	\textbf{Papagiannopoulos} & 8721 & - & 1794 \\
	AVR Crypto-lib & 105796 & 151624 & 1514 \\
	Eisenbarth & 10723 & 11239 & 936 \\
	\textbf{Verstegen} & & & \\
	\hspace{0.4em} Inlined rotation, unpacked S-Boxes (128-bit) &  64506 & 119626 & 292 \\
	\hspace{0.4em} Inlined rotation (128-bit)                   &  67854 & 123346 & 290 \\
	\hspace{0.4em} Inlined rotation, unpacked S-Boxes           &  52622 &  73952 & 280 \\
	\hspace{0.4em} Inlined rotation                             &  55784 &  77300 & 278 \\
	\hspace{0.4em} Unpacked S-Boxes                             & 205793 & 274832 & 276 \\
	\hspace{0.4em} Unpacked S-Boxes (128-bit)                   & 308445 & 631498 & 276 \\
	\hspace{0.4em} Default                                      & 208955 & 278180 & 274 \\
	\hspace{0.4em} Default (128-bit)                            & 311793 & 635218 & 274 \\
	\end{tabular}
	\label{numbers}

\end{table}
\end{frame}

\begin{frame}[fragile]
\frametitle{Relative Performance/Size}
\tiny{
\advance\leftskip-4em
\begin{gnuplot}[terminal=latex]
set terminal latex
set output "cycles_byte.tex"
set yrange [200:1900]
set xlabel "Cycles/byte"
set ylabel "Size"
set title "Efficiency vs Size"
plot "cycles_byte.txt" notitle with points, \
     ''               using 1:2:3 with labels left offset 1,0 notitle

\end{gnuplot}
\include{cycles_byte}
}
\end{frame}

\begin{frame}[fragile]
\frametitle{Binary QR Code}
\includegraphics[width=0.5\textwidth]{qr2}
%\includegraphics[width=0.5\textwidth]{qr3}

\textbf{Not} a URL - may crash QR apps
\end{frame}

\begin{frame}[fragile]
\frametitle{ASCII Art}
\scriptsize{
%\begin{snugshade}
    \begin{Verbatim}[commandchars=\\\{\}]
C56B90AD   3EF84712   5EF8C12    DB4630  79A57D0  3AD0    F1F  7F0E070E1
41D05DD05  CD047D080  2D16D00   82E81E1  06D0542  682E0   03D  04A9591F7
33C0CAE08  894CA9598  81991F9   883CD13  FACF9D1  E8A95   A9F  7089504D0
829   502  D08   295  089       5E8      2FE      F70E70  FE5     955
491  10F0  529   502  C00       0000     000      5F7080  7F8     52B
089587950  795879517  9587952   795879   5379508  9543958 6E0     D5D
F442687E3  D2DF802DD  DDF082E   4F31089  5CC278C  916 991 862     78D
93C830D1   F7A85008   9568E08     C91CD  DF8D936  A95 D9F7A85     008
954        427 F0E0   70E          0189  6DD      27C  C278D9     189
93C        A30 E1F7A  251      08   956  894      189  664E08     E91
CAD        FC9  DF6A  95D9F73  F932F931  F930F93  16F   4E894     F3C
F68        941   7966 4E08F91  8E93AA95  6A95D9F  71E   F4E89     419
96F        6CF   0895D7DFC5DF  CDDFE0D   FB7DFD9  F7C    0CF0     000
    \end{Verbatim}
%\end{snugshade}
}
\end{frame}

\begin{frame}[fragile]
\frametitle{ASCII Art}
\scriptsize{
%\begin{snugshade}
    \begin{Verbatim}[commandchars=\\\{\}]
s-boxes                                     decrypt (start+16)
|                                           |
\textcolor{black}{C56B90AD   3EF84712   5EF8C12    DB4630  79A}\textcolor{red}{57D0  3AD0    F1F  7F0E070E1}
\textcolor{red}{41D05DD05  CD047D080  2D16D00   82E81E1  06D0542  682E0   03D  04A9591F7}
\textcolor{red}{33C0}\textcolor{orange}{CAE08  894CA9598  81991F9   883CD13  FACF9D1  E8A95   A9F  70895}\textcolor{blue}{04D0}
\textcolor{blue}{829   502  D08   295  089       5}\textcolor{purple}{E8      2FE      F70E70  FE5     955}
\textcolor{purple}{491  10F0  529   502  C00       0000     000      5F7080  7F8     52B}
\textcolor{purple}{0895}\textcolor{brown}{87950  795879517  9587952   795879   5379508  95}\textcolor{cyan}{43958 6E0     D5D}
\textcolor{cyan}{F442687E3  D2DF802DD  DDF082E   4F31089  5}\textcolor{red}{CC278C  916 991 862     78D}
\textcolor{red}{93C830D1   F7A85008   95}\textcolor{magenta}{68E08     C91CD  DF8D936  A95 D9F7A85     008}
\textcolor{magenta}{95}\textcolor{pink}{4        427 F0E0   70E          0189  6DD      27C  C278D9     189}
\textcolor{pink}{93C        A30 E1F7A  251      08   95}\textcolor{yellow}{6  894      189  664E08     E91}
\textcolor{yellow}{CAD        FC9  DF6A  95D9F73  F932F931  F930F93  16F   4E894     F3C}
\textcolor{yellow}{F68        941   7966 4E08F91  8E93AA95  6A95D9F  71E   F4E89     419}
\textcolor{yellow}{96F        6CF   0895}\textcolor{green}{D7DFC5DF  CDDFE0D   FB7DFD9  F7C    0CF0     000}
                     |
                     encrypt (end-16)

    S-Boxes, \textcolor{red}{decrypt}, \textcolor{orange}{rotate_left_i}, \textcolor{blue}{sBoxByte}, \textcolor{purple}{sBoxNibble}, \textcolor{brown}{pLayerNibble},
    \textcolor{cyan}{schedule_key}, \textcolor{red}{addRoundKey}, \textcolor{magenta}{sBoxLayer}, \textcolor{pink}{setup}, \textcolor{yellow}{pLayer}, \textcolor{green}{encrypt}.
    \end{Verbatim}
%\end{snugshade}
}
\end{frame}

\begin{frame}[fragile]
\frametitle{Questions?}
	\footnotesize{
	\url{https://github.com/aczid/ru_crypto_engineering/}
	\url{https://github.com/kostaspap88/PRESENT_speed_implementation/}
	}
	\includegraphics[width=0.5\textwidth]{qr_url}
	\includegraphics[width=0.5\textwidth]{qr_url2}
\end{frame}

\end{document}
