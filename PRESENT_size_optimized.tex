\documentclass{beamer}
\usepackage{listings}
\lstset{basicstyle=\scriptsize\sffamily}
%\lstset{language=Assembler}[avr8avra] % not standard
\lstdefinestyle{customasm}{
  morekeywords={adc,adiw,asr,brts,brcs,brhs,brne,cbr,clh,clr,clt,cp,cpi,dec,eor,ldi,lpm,lsl,mov,or,rcall,ret,seh,set,subi,swap,rjmp},
  morecomment=[l][\color{blue}]{;}
}
\lstset{style=customasm}

\begin{document}

\title{A size-optimized implementation of PRESENT for AVR machines}
\author{
Aram Verstegen \\
Kostas Papagiannopoulos
}
\date{\today}

\frame{\titlepage}

\begin{frame}
\frametitle{AVR architecture}
        \begin{itemize}
        \item 16-bit opcodes
        \item Harvard architecture
        \item Some SRAM
        \item No bulk instructions
        \item SREG flags
        \item SWAP instruction
        \item CBR instruction
        \item All kinds of branching instructions
        \end{itemize}
\end{frame}

\begin{frame}
\frametitle{Optimization strategy}
        \begin{itemize}
        \item Be concise to start with
        \item Look for repeating patterns
        \item Refactor to reuse code wherever possible
        \item Use the SRAM more
        \item Employ a different I/O pattern
        \end{itemize}
\end{frame}

\begin{frame}[fragile]
\frametitle{Serialization of the algorithm}
\begin{lstlisting}
; state ^= roundkey (first 8 bytes of key register)
addRoundKey:
        eor STATE0, KEY0
        eor STATE1, KEY1
        eor STATE2, KEY2
        eor STATE3, KEY3
        eor STATE4, KEY4
        eor STATE5, KEY5
        eor STATE6, KEY6
        eor STATE7, KEY7
        ret
\end{lstlisting}
\end{frame}

\begin{frame}[fragile]
\frametitle{Serialization of the algorithm}
\begin{lstlisting}
; state ^= roundkey (first 4 bytes of key register)
addRoundKey:
        eor STATE0, KEY0
        eor STATE1, KEY1
        eor STATE2, KEY2
        eor STATE3, KEY3
        ret
\end{lstlisting}

This helps with:
\begin{itemize}
        \item doing I/O
        \item applying round keys
        \item applying S-Boxes
        \item applying P-layer
\end{itemize}

\end{frame}

\begin{frame}[fragile]
\frametitle{Using SREG flags}
\begin{lstlisting}
setup_redo_block:
        clt                   ; clear T flag
        rjmp redo_block       ; do the second part
block:
        set                   ; set T flag
        ; fall through
redo_block:
        ; instructions here happen twice when called from block

        brts setup_redo_block ; redo this block? (if T flag set)
        ret
\end{lstlisting}
\end{frame}

\begin{frame}[fragile]
\frametitle{Packing S-Boxes}
	\footnotesize{
	Before: \\
	\begin{tabular}{ | c | c | c | c | c | c | c | c | c | c | c | c | c | c | c | c | }
	  \hline                        
	  C & 5 & 6 & B & 9 & 0 & A & D & 3 & E & F & 8 & 4 & 7 & 1 & 2 \\
	  \hline  
	\end{tabular}
	\\

	After: \\
	\begin{tabular}{ | c | c | c | c | c | c | c | c | }
	  \hline                        
	  C5 & 6B & 90 & AD & 3E & F8 & 47 & 12 \\
	  \hline  
	\end{tabular}
	}

\begin{lstlisting}
unpack_sBox:
        asr ZL                ; halve input, take carry
        lpm SBOX_OUTPUT, Z    ; get s-box output
        brcs odd_unpack       ; branch depending on carry
even_unpack:
        swap SBOX_OUTPUT      ; swap nibbles in s-box output
odd_unpack:
        cbr SBOX_OUTPUT, 0xf0 ; clear high nibble in s-box output
        ret
\end{lstlisting}
\end{frame}

\begin{frame}[fragile]
\frametitle{S-Box optimization}
\begin{lstlisting}
sBoxByte:
        ; input (low nibble)
        mov ZL, ITEMP     ; load s-box input
        rcall unpack_sBox ; get output in SBOX_OUTPUT

        cbr ITEMP, 0xf    ; clear low nibble in output register
        or ITEMP, TEMP0   ; save low nibble to output register

        ; fall through
sBoxHighNibble:
        mov ZL, ITEMP     ; load s-box input
        rcall unpack_sBox ; get output in SBOX_OUTPUT
        swap SBOX_OUTPUT  ; move low nibble to high nibble

        cbr ITEMP, 0xf0   ; clear high nibble in output
        or ITEMP, TEMP0   ; save high nibble to output

        ret
\end{lstlisting}
\end{frame}

\begin{frame}[fragile]
\frametitle{S-Box optimization}
\begin{lstlisting}
sBoxHighNibble:
        clh
        swap ITEMP            ; swap nibbles
        rjmp sBoxLowNibble    ; do low nibble
sBoxByte:
        seh
        ; fall through
sBoxLowNibble:
        mov ZL, ITEMP         ; load s-box input
        cbr ZL, 0xf0          ; clear high nibble in s-box input

        rcall unpack_sBox

        cbr ITEMP, 0xf        ; clear low nibble in IO register
        or ITEMP, SBOX_OUTPUT ; save low nibble to IO register
        brhs sBoxHighNibble   ; do the high nibble
        swap ITEMP            ; swap nibbles
        ret
\end{lstlisting}
\end{frame}

\begin{frame}[fragile]
\frametitle{S-Box optimization}
\begin{lstlisting}
sBoxHighNibble:
        clh
        swap ITEMP            ; swap nibbles
        rjmp sBoxLowNibble    ; do low nibble
sBoxByte:
        seh                   ; set H flag to re-do this block
sBoxLowNibble:
        mov ZL, ITEMP         ; load s-box input from IO register
        cbr ZL, 0xf0          ; clear high nibble in s-box input

unpack_sBox:
        asr ZL                ; halve input, take carry
        lpm SBOX_OUTPUT, Z    ; get s-box output
        brcs odd_unpack       ; branch depending on carry
even_unpack:
        swap SBOX_OUTPUT      ; swap nibbles in s-box output
odd_unpack:
        cbr SBOX_OUTPUT, 0xf0 ; clear high nibble in s-box output

        cbr ITEMP, 0xf        ; clear low nibble in IO register
        or ITEMP, SBOX_OUTPUT ; save low nibble to IO register
        brhs sBoxHighNibble   ; do the high nibble
        swap ITEMP            ; swap nibbles
        ret
\end{lstlisting}
\end{frame}

\lstset{basicstyle=\tiny\sffamily}
\begin{frame}[fragile]
\frametitle{Code}
	\begin{tabular}{ l r }
\begin{lstlisting}
encrypt:
  rcall setup
  subi XL, 8
  encrypt_update:
    rcall roundkey_ram
    rcall SPnet
    rcall consecutive_input
    rcall interleaved_output
    ldi ITEMP, 32
    rcall rotate_left_i
    rcall addRoundkey
    rcall state_to_output
    rcall SPnet
    adiw XL, 9
    rcall interleaved_output
    dec XL
    ldi ITEMP, 48
    rcall rotate_left_i
    rcall schedule_key
    cpi ROUND_COUNTER, 32
    brne encrypt_update
  rcall last_round_key
  ret
\end{lstlisting}
&
\begin{lstlisting}
decrypt:
  rcall setup
  schedule_last_key:
    rcall schedule_key
    cpi ROUND_COUNTER, 32
    brne schedule_last_key
  ldi ZH, high(INVSBOX<<1)
  decrypt_update:
    subi XL, 8
    rcall last_round_key
    rcall interleaved_input
    rcall invSPnet
    adiw XL, 9
    rcall interleaved_input
    dec XL
    rcall consecutive_output
    rcall invSPnet
    rcall consecutive_output
    inv_schedule_key:
      dec ROUND_COUNTER
      mov ITEMP, KEY6
      rcall sBoxHighNibble
      mov KEY6, ITEMP
      ldi ITEMP, 1
      rcall rotate_left_i
      eor KEY3, ROUND_COUNTER
      ldi ITEMP, 66
      rcall rotate_left_i
    cpi ROUND_COUNTER, 1
    brne decrypt_update
  subi XL, 8
  rcall last_round_key
  ret
\end{lstlisting}
	\end{tabular}
\end{frame}


\begin{frame}[fragile]
\frametitle{Numbers}
	\begin{tabular}{ l c c }

                & Speed optimized & Size optimized \\
        Size (bytes) & 1794 & 428 \\
        Cycles (encryption) & 8721 & 91390 \\
        Cycles (decryption) & - & 104937 \\
        RAM (bytes) & 18 & 18

	\end{tabular}
\end{frame}

\begin{frame}[fragile]
\frametitle{Questions?}
	\footnotesize{
	\url{https://github.com/aczid/ru_crypto_engineering}
	\url{https://github.com/kostaspap88/PRESENT_speed_implementation/}
	}
\end{frame}
\end{document}
